\newpage
\section{Introduction}

The University Management System (UMS) was designed by four computer engineering students within the course ENGG 1420: Object-Oriented programming in Java. This project aims to develop both our technical, project management, teamwork, and communication skills to deliver a fully functional University of Guelph Management System. This software system leverages an excel file for storage and data warehousing and management that is integrated with the system. For example, add and remove features that seamlessly coincide with the excel file and the software.



\subsection{Purpose of the System}

The goal of UMS is to provide users and administrators with the ability to view and interact with the courses offered at the University of Guelph using a simple and easy-to-use interface. The University Management systems focus on a seamless integration of easy-to-understand commands and features that allow users to ensure that they can remain organized and accountable on the courses and changes that are provided to their user interface.

\subsection{Key Features}

The University Management System contains many features to name a few, which are highlighted as the following.
\begin{itemize}
    \item \textbf{User (Student):} Full integration of with Data Archive Excel
    \item \textbf{User (Student):} Detail Course Overview
    \item \textbf{User (Student):} Information highlights
    \item \textbf{User (Student):} Profile Picture Changer
    \item \textbf{User (Faculty):} Student Enrolment viewer
    \item \textbf{User (Faculty):} Personal Faculty Information
    \item \textbf{User (Administrator):} Add/Remove Subjects
    \item \textbf{User (Administrator):} Add/Remove Courses
    \item \textbf{User (Administrator):} Add/Remove Students
    \item \textbf{User (Administrator):} Add/Remove Events
    
\end{itemize}

\subsection{Role-Based Access Overview (Admin vs. User)}

The University Management System contains two key role-based features that require the correct log-in security protocols. First, the user consisting of the student and faculty members. These users have a specific set of rules and privileges that they are allowed to do within the UMS. These rules do not apply to the privileges of an administrator. Second, the administrator has full access to all the rules allowing for functional and technical changes are required. Both the user and the administrator have the same dashboards, but access and functionalities are different. Throughout the user guide, it aims to provide the difference between these two roles, including their limitations.



