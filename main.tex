%----------------------------------------------------------------------------------------
%	PACKAGES AND DOCUMENT CONFIGURATIONS
%----------------------------------------------------------------------------------------

\documentclass[12pt]{article}

% Adjusting margins to personal my need
\addtolength{\oddsidemargin}{-.5in}
\addtolength{\evensidemargin}{-.5in}
\addtolength{\textwidth}{1in}
\addtolength{\topmargin}{-.5in}
\addtolength{\textheight}{1in}

% Graphics
\usepackage{caption}
\usepackage{subcaption}
\usepackage{graphicx}
\graphicspath{{figures/}}
\usepackage{enumitem}
\usepackage{float}

% Math
\usepackage{amssymb}
\usepackage{amsmath} % Required for some math elements 

\usepackage{listings}
\usepackage[framed,numbered,autolinebreaks,useliterate]{mcode}



% Other
\usepackage{comment}
\usepackage{algorithmic}
\usepackage{array}
\usepackage{lipsum}
\usepackage{hyperref}
\usepackage{indentfirst}


\renewcommand{\refname}{Resources} % For articles



%----------------------------------------------------------------------------------------
%	MAIN PART
%----------------------------------------------------------------------------------------
\begin{document}

\title{\textbf{ENGG 1420 Project} \\ University Management System \\ User Guide} % Title
%\author{Konstantin Akhmadeev, That Guy, Yet Another Guy (in alphabetic order)}
\author{\textbf{Tevin Heath, B.Eng (2028)}\\ \href{mailto:heatht@uoguelph.ca}{heatht@uoguelph.ca} \\\\  \textbf{Ahmad Alwan, B.Eng (2029)}\\ \href{mailto:aalwan@uoguelph.ca}{aalwan@uoguelph.ca} \\\\  \textbf{Daniel Clarke, B.Eng (2029)}\\\href{mailto:dclark28@uoguelph.ca}{dclark28@uoguelph.ca} \\\\  \textbf{Ahmed Saleem, B.Eng (2029)}\\ \href{mailto:asalee07@uoguelph.ca}{asalee07@uoguelph.ca} }
% \date{\today} % Date for the report
\maketitle % Inserts the title, author and date
\thispagestyle{empty}

\begin{figure}[H] % [H] forces the figure to be placed exactly here
    \centering
    \includegraphics[width=0.5\linewidth]{figures/SOE Lockup WEB - LARGE - transparent bkgd - RED ENG.png}
    \label{fig:enter-label}
\end{figure}

% Add the Supervisor section below the title page
\vspace{2mm} % Adds some vertical space between the title and the supervisor section
% \begin{abstract}
% %% Text of the abstract
% \end{abstract}



\newpage
\tableofcontents


% \begin{abstract}
% %% Text of the abstract
% \noindent This report highlights the usage of successive parabolic interpolation and aims to helping readers understand and implement C++ and MATLAB code to test their experiment. This reports assist in providing exercises and a discussion on areas of engineering that this topic can be applied.


% % This template will help you to write your bibliographic and final reports using \LaTeX{}. You'll find here the examples of text structuring as well as tables, figures, citations and references. For other features of \LaTeX, see tutorials on \href{https://www.overleaf.com/learn}{\textbf{Overleaf}} or use this \href{https://wch.github.io/latexsheet/}{\textbf{cheatsheet}}. To work with this template, download its entire folder (including /bibliography and /figures), and run your \LaTeX{}  editor like \href{http://www.xm1math.net/texmaker/}{\textbf{Texmaker}} or \href{https://www.overleaf.com}{\textbf{Overleaf}}. Then make a plan by changing the document's structure with \textit{section} and \textit{subsection} commands. Finally, delete the \textit{lipsum} fillings and start writing you report. Good luck!
% \end{abstract}

\newpage
\newpage
\section{Introduction}

The University Management System (UMS) was designed by four computer engineering students within the course ENGG 1420: Object-Oriented programming in Java. This project aims to develop both our technical, project management, teamwork, and communication skills to deliver a fully functional University of Guelph Management System. This software system leverages an excel file for storage and data warehousing and management that is integrated with the system. For example, add and remove features that seamlessly coincide with the excel file and the software.



\subsection{Purpose of the System}

The goal of UMS is to provide users and administrators with the ability to view and interact with the courses offered at the University of Guelph using a simple and easy-to-use interface. The University Management systems focus on a seamless integration of easy-to-understand commands and features that allow users to ensure that they can remain organized and accountable on the courses and changes that are provided to their user interface.

\subsection{Key Features}

The University Management System contains many features to name a few, which are highlighted as the following.
\begin{itemize}
    \item \textbf{User (Student):} Full integration of with Data Archive Excel
    \item \textbf{User (Student):} Detail Course Overview
    \item \textbf{User (Student):} Information highlights
    \item \textbf{User (Student):} Profile Picture Changer
    \item \textbf{User (Faculty):} Student Enrolment viewer
    \item \textbf{User (Faculty):} Personal Faculty Information
    \item \textbf{User (Administrator):} Add/Remove Subjects
    \item \textbf{User (Administrator):} Add/Remove Courses
    \item \textbf{User (Administrator):} Add/Remove Students
    \item \textbf{User (Administrator):} Add/Remove Events
    
\end{itemize}

\subsection{Role-Based Access Overview (Admin vs. User)}

The University Management System contains two key role-based features that require the correct log-in security protocols. First, the user consisting of the student and faculty members. These users have a specific set of rules and privileges that they are allowed to do within the UMS. These rules do not apply to the privileges of an administrator. Second, the administrator has full access to all the rules allowing for functional and technical changes are required. Both the user and the administrator have the same dashboards, but access and functionalities are different. Throughout the user guide, it aims to provide the difference between these two roles, including their limitations.




\newpage
\section{Getting Started}
The University Management systems has been created using Java, this means that in order to run the program, there needs to exist a class that allows for the program for the program to run. In the UMS, we designed the loginScreen class as the program to run, once the user runs the loginScreen class it will begin the program.

The ability to run the software requires the installation of IntellJ or Replit, or another IDE that allows Java programs to run functionality. It is important to note that the University Management System is subject to version changes or updates to Java. For more details relating to the code, refer to the Class Library for each of the classes that were designed. Once the user runs the program through IntellJ by running the Login Screen class, begin by entering the correct credentials provided by the University of Guelph, see \autoref{fig:example1}


\begin{figure}[ht]
    \centering
        \centering\includegraphics[width=1\linewidth]{figures/login_screen.png}
        \caption{Login Screen}
        \label{fig:example1}  
\end{figure}

% \subsection{Password Recovery (Optional)}


\subsection{First-Time Login Guide}

Once the user is at the login screen after running the program, they have either three login options to utilize the UMS program; faculty, student, and administrator. Each of these log-in credentials provides specific privileges that allow for editing such as deleting or adding, viewing, or changing personal settings. First time users must provide their correct login information in order to access the system, if they enter either an incorrect password or incorrect username, the user will receive an error message (see \autoref{fig:example2} below). This error will continue to occur until the user places the correct information, the user must login with the correct username and password; otherwise, it will continue to provide an error.


\begin{figure}[ht]
    \centering
        \centering\includegraphics[width=1\linewidth]{figures/Incorrect_PW.png}
        \caption{Invalid Username/Password}
        \label{fig:example2}  
\end{figure}
\newpage
\section{Dashboard \& Additional Features Overview}
The University Management System (UMS) contains a dashboard as shown in \autoref{fig:example3} and initial features upon logging in with the correct credentials. First, the additional features once a user logs in are the collapse menu button and the logout button. Both of these buttons are consistent across all types of users (e.g., student, faculty of administrators). The collapse menu allows the user to hide the navigation panel or keep it open, and the logout button once clicked places the user back into the login page at the start of the program. In the upper right corner, the UMS has the feature to maximize, minimize, and close the software. Each of these buttons is designed to ensure that the software integration remains seamless once pressed.


The UMS also contains a viewable dashboard after logging in, providing key statistics and relevant numbers for quick understanding. The dashboard, as shown in \autoref{fig:example3}, is the initial page that provides an overview. Provides an overview of all students, courses, faculty, and events within the University.


\begin{figure}[ht]
    \centering
        \centering\includegraphics[width=0.7\linewidth]{figures/Dashboard.png}
        \caption{Dashboard}
        \label{fig:example3}  
\end{figure}



\newpage
\subsection{Navigation Menu}
The University Management System (UMS) contains a navigation panel on the left side of the program once the correct user logins. These have the following tabs; Dashboard, Subject Management, Course Management, Student Management, Faculty Management, and Event Management, as shown in \autoref{fig:example4}. Each of these tabs are clickable, allowing the user to move throughout each one. More details on each page are provided in subsequent sections. 

\begin{itemize}
    \item Subject Management: Overview of Subjects at the University.
    \item Course Management: Overview of all the courses, subject code, capacity, lecture time, exam, location and teacher offered at the University.
    \item Student Management: Overview of the Student at the University.
    \item Faculty Management: Overview of the Faculty at the University.
    \item Event Management: Schedule of all the events at the University.
\end{itemize}

\begin{figure}[h!]
    \centering
    \adjustbox{max width=0.125\linewidth}{\includegraphics{figures/Navigation_Menu.png}}
    \caption{Navigation Panel}
    \label{fig:example4}
\end{figure}
\FloatBarrier


% \begin{figure}[h!]
%     \centering
%     \includegraphics[width=0.45\textwidth]{figures/Navigation_Menu.png}
%     \caption{Dashboard}
% \end{figure}
% \FloatBarrier
\newpage
\section{Admin-Specific Features}

\subsection{User Management}

\subsubsection{Adding/Editing Users}

\subsubsection{Assigning Roles}

\subsection{Course Management}

\subsubsection{Creating/Updating Courses}

\subsubsection{Enrolling Students}

\subsection{Faculty Management}

\subsubsection{Assigning Faculty to Courses}
\newpage
\section{User-Specific Features (Students/Faculty)}

\subsection{Viewing Personal Information}

\subsection{Accessing Course Details}

\subsection{Checking Grades (Students)}

\subsection{Viewing Assigned Classes (Faculty)}


\newpage
\section{Troubleshooting \& Support}

Throughout the design of this program, there may be bugs that might have been an oversight. Troubleshooting continues to become a problem and in the test that was conducted for user experience, the main error issue that exists is logging into the UMS.  These bugs may require troubleshooting to fix the error (e.g., spelling or capitalization for logging in). Although the program aims to support and be bug-free, they might remain in the program and if they are persistent and affecting the experience. Contact the authors of the program for assistance.



\subsection{Common Login Issues}


\subsection{Frequently Asked Question}

% \subsection{Error Message \& Solutions (Optional)}
\newpage
\section{Appendix}


\subsection{Glossary of Terms}

\begin{lstlisting}[language=Java]
package com.example.university_management;

import org.apache.poi.ss.usermodel.*;
import java.io.File;
import java.io.FileInputStream;
import java.io.IOException;

public class ReadingFaculties {
    private static final String FILE_PATH = loginController.filePath;
    private static Faculties[] faculties;
    public static void loadFaculties() throws IOException {
        FileInputStream fins = new FileInputStream(new File(FILE_PATH));
        Workbook wb = WorkbookFactory.create(fins);
        Sheet sheet = wb.getSheetAt(3);
        int rowCount = 0;
        for (int i = 0; i <= sheet.getLastRowNum(); i++) {
            Row row = sheet.getRow(i);
            if (row != null && row.getPhysicalNumberOfCells() > 0) { // Check if the row has actual cells
                rowCount++;
            }
        }
        faculties = new Faculties[rowCount];
        for (int i = 1; i <= rowCount; i++) { // Start from row 1 (skip header)
            Row row = sheet.getRow(i);
            if (row == null) continue;
            faculties[i-1] = new Faculties(getCellValue(row.getCell(0)), getCellValue(row.getCell(1)), getCellValue(row.getCell(2)), getCellValue(row.getCell(3)),
                    getCellValue(row.getCell(4)), getCellValue(row.getCell(5)), getCellValue(row.getCell(6)), getCellValue(row.getCell(7)));
        }
        wb.close();  // Close workbook
        fins.close();
    }
    static String getCellValue(Cell cell) {
        if (cell == null) return ""; // Return empty string for null cells

        return switch (cell.getCellType()) {
            case STRING -> cell.getStringCellValue().trim(); // Return string value
            case NUMERIC -> String.valueOf(cell.getNumericCellValue()).trim(); // Convert numeric value to string
            case BOOLEAN -> String.valueOf(cell.getBooleanCellValue()).trim(); // Convert boolean to string
            default -> ""; // Return empty string for other cell types
        };
    }
    public static boolean verifyLogin(String enteredID, String enteredPass) throws IOException {
        loadFaculties(); // Load student data
        if (enteredID.isEmpty() || enteredPass.isEmpty()) return false; // Reject empty credentials

        // Loop through students to check for a matching ID and password
        for (Faculties faculties1 : faculties) {
            if(faculties1==null) continue;
            if (enteredID.equals(faculties1.ID) && enteredPass.equals(faculties1.password)) {
                return true; // Login successful
            }
        }
        return false; // Login failed
    }
    public static Faculties facultiesInfo(String userID) throws IOException{
        for (Faculties faculties1: faculties){
            if (userID.equals(faculties1.ID)){
                return new Faculties(faculties1.ID, faculties1.name, faculties1.degree, faculties1.researchInterest, faculties1.email, faculties1.officeLocation,
                        faculties1.coursesOffered, faculties1.password);
            }
        }
        return null;
    }
    public static Faculties[] getAllFaculty(){
        return faculties;
    }
}







\end{lstlisting}

% \begin{figure}[ht]
%     \centering
%         \centering\includegraphics[width=1\linewidth]{placeholder}
%         \caption{An example of multiple figures in one frame.}
%         \label{fig:example11}  
% \end{figure}

\lipsum[9]


% \section{Results (optional)}

% \begin{figure}[ht]
%     \centering
%     \begin{subfigure}[t]{0.4\textwidth}
%         \centering\includegraphics[width=1\linewidth]{placeholder}
%         \caption{An example of multiple figures in one frame.}
%         \label{fig:example11}
%     \end{subfigure}
%     %
%     \begin{subfigure}[t]{0.4\textwidth}
%         \centering\includegraphics[width=1\linewidth]{placeholder}
%         \caption{Next subfigure.}
%         \label{fig:example12}
%     \end{subfigure}
%     %
%     \\
%     \begin{subfigure}[t]{0.4\textwidth}
%         \centering\includegraphics[width=1\linewidth]{placeholder}
%         \caption{Subfigure on another line.}
%         \label{fig:example21}
%     \end{subfigure}
%     %
%     \begin{subfigure}[t]{0.4\textwidth}
%         \centering\includegraphics[width=1\linewidth]{placeholder}
%         \caption{Yet another subfigure.}
%         \label{fig:example22}
%     \end{subfigure}    
% \end{figure}

% \lipsum[9]





%----------------------------------------------------------------------------------------
%	Bibliography
% %----------------------------------------------------------------------------------------
% \bibliography{bibliography/sample}{}
% \bibliographystyle{plain}

\end{document}